\documentclass[a4paper,12pt]{report}
\usepackage[utf8]{vietnam}
\usepackage{amsmath}
\usepackage{amsfonts}
\usepackage{enumitem}
%\usepackage{amssymb}
\usepackage{graphicx}
%\usepackage{cases}
\usepackage{fancybox}
\usepackage{multirow}
\usepackage{longtable}
\usepackage{listings}
\usepackage[nottoc]{tocbibind}
\usepackage{indentfirst}
\usepackage[english]{babel}
\usepackage{float}
\PassOptionsToPackage{hyphens}{url}\usepackage{hyperref}  
\usepackage[left=3cm, right=2.00cm, top=2.00cm, bottom=2.00cm]{geometry}
%\lstset{
   %keywords={break,case,catch,continue,else,elseif,end,for,function,
   %   global,if,otherwise,persistent,return,switch,try,while},
%   language = Java,
%   basicstyle=\ttfamily \fontsize{12}{15}\selectfont,   
	% numbers=left,
%   frame=lrtb,
%tabsize=3
%}
\hypersetup{
    colorlinks,
    citecolor=black,
    filecolor=black,
    linkcolor=blue,
    urlcolor=red 
}
\setlength{\parskip}{0.6em}
\addto\captionsenglish{%
 \renewcommand\chaptername{Phần}
 \renewcommand{\contentsname}{Mục lục} 
 \renewcommand{\listtablename}{Danh sách bảng}
 \renewcommand{\listfigurename}{Danh sách hình vẽ}
 \renewcommand{\tablename}{Bảng}
 \renewcommand{\figurename}{Hình}
 \renewcommand{\bibname}{Tài liệu tham khảo}
}

%\newtheorem{definition}{Định nghĩa}[chapter]
%\newtheorem{lema}{Bổ đề}[chapter]
%\newtheorem{theorem}{Định lý}[chapter]

\begin{document}
\thispagestyle{empty}
\thisfancypage{
\setlength{\fboxrule}{1pt}
\doublebox}{}

\begin{center}
{\fontsize{16}{19}\fontfamily{cmr}\selectfont TRƯỜNG ĐẠI HỌC BÁCH KHOA HÀ NỘI\\
VIỆN CÔNG NGHỆ THÔNG TIN VÀ TRUYỀN THÔNG}\\
\textbf{------------*******---------------}\\[1cm]
\includegraphics[scale=0.13]{hust.jpg}\\[1.3cm]
{\fontsize{32}{43}\fontfamily{cmr}\selectfont BÁO CÁO}\\[0.1cm]
{\fontsize{38}{45}\fontfamily{cmr}\fontseries{b}\selectfont MÔN HỌC}\\[0.2cm]
{\fontsize{20}{24}\fontfamily{phv}\selectfont Các thuật toán cơ bản trong tính toán tiến hoá}\\[0.3cm]
{\fontsize{18}{20}\fontfamily{cmr}\selectfont \emph{Đề tài: Tiến hóa đa nhiệm }}\\[3cm]
\hspace{-5cm}\fontsize{14}{16}\fontfamily{cmr}\selectfont \textbf{Nhóm sinh viên thực hiện:}\\[0.1cm] 
\begin{longtable}{l c c}
Nguyễn Tuấn Đạt & 20130856 & CNTT2.02-K58 \\
Phan Anh Tú &   20134501 & CNTT2.01-K58\\
\end{longtable}
\vspace{0.5cm}
\hspace{-6cm}\fontsize{14}{16}\fontfamily{cmr}\selectfont \textbf{Giảng viên hướng dẫn:}\\[0.1cm]
\hspace{-2.7cm}\fontsize{14}{16}\fontfamily{cmr}\selectfont PGS-TS.Huỳnh Thị Thanh Bình \\[3cm]
\fontsize{16}{19}\fontfamily{cmr}\selectfont Hà Nội 5--2017
\end{center}
\newpage
\pdfbookmark{\contentsname}{toc}
\tableofcontents
%\listoftables
%\listoffigures

\chapter{Cơ sở lý thuyết}


\section{Giải thuật tiến hóa}
Giải thuật tiến hóa là thuật toán tối ưu meta-heuristics dựa trên thuyết tiến hóa (chọn lọc tự nhiên) của Dac-uyn.
\par Thuật toán bắt đầu với quần thể gồm nhiều cá thể áp dụng các toán tử đột biến trên một vài cá thể và các toán tử lai ghép giữa các cá thể với nhau để tạo ra thế hệ mới với mong muốn thế hệ sau sẽ tốt hơn thế hệ trước. Thuật toán sẽ lặp đi lặp lại quá trình chọn lọc (tạo thế hệ mới) với mong muốn sẽ chọn được một cá thể tốt nhất trong một thế hệ nào đó đồng nghĩa bài toán tối ưu đầu vào sẽ có đạt được giá trị tối ưu.
\par Để có thể cài đặt thuật toán ta cần phải mã hóa mỗi cá thể theo một cấu trúc nào đó để máy tính có thể hiểu được, thêm vào đó cần phải định nghĩa các toán tử đột biến và lai ghép để có thể thực thi trên máy. Trong phần tiếp theo chúng em sẽ trình bày một số cách mã hóa và một số toán tử đột biến và lai ghép thông dụng.

\subsection{Một số cách mã hóa}


\subsection{Một số toán tử đột biến}

\subsection{Một số toán tử lai ghép}

 
\section{Tiến hóa đa nhiệm}
Các bài toán tối ưu thông thường được phân vào 2 nhóm: tối ưu hóa một hàm mục tiêu và tối ưu hóa nhiều hàm mục tiêu. 


\chapter{Ứng dung}
\section{Bài toán ứng dụng}
\section{Kết quả thực tế}
\subsection{TSP-KP}
\subsection{TSP-KP Kết hợp tích chập}



\begin{thebibliography}{9}
\end{thebibliography}
\end{document}
